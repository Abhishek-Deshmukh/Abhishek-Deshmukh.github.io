\documentclass[a4paper, 11pt]{proc}
\begin{document}
\title{Fashion Show - Umang}
\author{Rule Book}
\maketitle
\begin{enumerate}
	\item All the Int. MSc. and PhD. students can participate in the event.
	\item Each contestant will register individually.
	\item Contestants are required to bring their own Props, Costumes and Make-up. None will be provided by the organizers.
	\item The decision of the judges shall be final and binding to all.
	\item The order of appearance will be determined by chit system.
	\item Organisers are not responsible for the insurance of the contestants.
	\item The organising committee reserves the right to make changes to these rules.
	\item First rounds for both male and female will consist of ramp walk, posing and a 1 min to 3 min speech about them and their costumes. Each candidtae will be given a maximum of 4 min for this round.
	\item If there are less than 5 contestants in any category, then the categories will be merged.
	\item During the first round/s the contestants will be scored by the judges based on their Confidence, Costume, Walking Style, and the Speech.
	\item Round 1 is an elimination round.
	\item Contestants who reach round 2 will be paired up with contestants from all categories based on chit system.
	\item In Round 2 contestants have to do a pair ramp walk with their designated partners and pose as a pair. The pair can have multiple poses.
	\item In round 2 contestants will be judged by their ability to pair up and pose for a picture.
	\item Contestants will be provided with rooms before rounds for preparation.
	\item Will be 2 prizes:- winner(male) and winner(female).
	\item The contestants can take help from other non-participating students of NISER for preparing their costumes. In such cases, it is contestant's responsibility to mention their names if they are awarded.
	\item Theme for the event: Fusion of Indian-regional dress with other dress from foreign countries.Ex:You can fuse a Chinese and a Japanese dress with a north eastern regional dress.
	\item The Judges are Akhil Ranjan, software engineer and photographer and Ankita Mishra, Law professional and Actor.
\end{enumerate}
Event Coordinators:- Dhairya Patel and Abhishek Anil Deshmukh

For Queries about the theme contact: Payal Priyadasani-7788809128
\end{document}
